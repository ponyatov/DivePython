\documentclass[oneside,12pt]{book}

% e-book format
\usepackage[paperwidth=210mm,paperheight=148mm,margin=10mm]{geometry}

% Cyrillization
\usepackage[T1,T2A]{fontenc}
\usepackage[utf8]{inputenc}
\usepackage[english,russian]{babel}
\usepackage{indentfirst}

% pdflatex options
\usepackage[unicode]{hyperref}
\usepackage[pdftex]{graphicx}

% listings
\usepackage{listings}
\lstset{
extendedchars=true,inputencoding=utf8, % i18n
frame=single, % show frames around
numbers=left, numberstyle=\small,numbersep=1mm,% line numbering
tabsize=4, % tab style
% keywordstyle=\color{blue}\texttt,
% commentstyle=\color{magenta}\texttt,
% showspaces=false
}

% misc
\newcommand{\email}[1]{$<$\href{mailto:#1}{#1}$>$}

\author{Mark Pilgrim \email{mark@diveintopython.org}}
\title{Dive Into Python}
\date{20 May 2004: Версия 5.4}

\begin{document}

\maketitle

\begin{titlepage}

Dive Into Python
\bigskip

20 May 2004
\bigskip

Copyright (\copyright) 2000, 2001, 2002, 2003, 2004 Mark Pilgrim (
\email{mark@diveintopython.org})
\bigskip

This book lives at \url{http://diveintopython.org/}. If you're reading it
somewhere else, you may not have the latest version.
\bigskip

Permission is granted to copy, distribute, and/or modify this document under
the terms of the GNU Free Documentation License, Version 1.1 or any later
version published by the Free Software Foundation; with no Invariant Sections,
no Front-Cover Texts, and no Back-Cover Texts. A copy of the license is
included in Appendix G, GNU Free Documentation License.
\bigskip

The example programs in this book are free software; you can redistribute 
and/or modify them under the terms of the Python license as published by the Python
Software Foundation. A copy of the license is included in Appendix H, Python
license.

\end{titlepage}

\tableofcontents
% \input{preface}

\chapter{1. Installing Python}
\section{1.1. Which Python is right for you?}
\section{1.2. Python on Windows}
\section{1.3. Python on Mac OS X}
\section{1.4. Python on Mac OS 9}
\section{1.5. Python on RedHat Linux}
\section{1.6. Python on Debian GNU/Linux}
\section{1.7. Python Installation from Source}
\section{1.8. The Interactive Shell}
\section{1.9. Summary}
   
\chapter{2. Your First Python Program}
\section{2.1. Diving in}
\section{2.2. Declaring Functions}
\subsection{2.2.1. How Python's Datatypes Compare to Other Programming
Languages}
       
\section{2.3. Documenting Functions}
\section{2.4. Everything Is an Object}
\subsection{2.4.1. The Import Search Path}
\subsection{2.4.2. What's an Object?}
       
\section{2.5. Indenting Code}
\section{2.6. Testing Modules}
   
\chapter{3. Native Datatypes}
\section{3.1. Introducing Dictionaries}
\subsection{3.1.1. Defining Dictionaries}
\subsection{3.1.2. Modifying Dictionaries}
\subsection{3.1.3. Deleting Items From Dictionaries}
       
\section{3.2. Introducing Lists}
\subsection{3.2.1. Defining Lists}
\subsection{3.2.2. Adding Elements to Lists}
\subsection{3.2.3. Searching Lists}
\subsection{3.2.4. Deleting List Elements}
\subsection{3.2.5. Using List Operators}
       
\section{3.3. Introducing Tuples}
\section{3.4. Declaring variables}
\subsection{3.4.1. Referencing Variables}
\subsection{3.4.2. Assigning Multiple Values at Once}
       
\section{3.5. Formatting Strings}
\section{3.6. Mapping Lists}
\section{3.7. Joining Lists and Splitting Strings}
\subsection{3.7.1. Historical Note on String Methods}
       
\section{3.8. Summary}
   
\chapter{4. The Power Of Introspection}
\section{4.1. Diving In}
\section{4.2. Using Optional and Named Arguments}
\section{4.3. Using type, str, dir, and Other Built-In Functions}
\subsection{4.3.1. The type Function}
\subsection{4.3.2. The str Function}
\subsection{4.3.3. Built-In Functions}
       
\section{4.4. Getting Object References With getattr}
\subsection{4.4.1. getattr with Modules}
\subsection{4.4.2. getattr As a Dispatcher}
       
\section{4.5. Filtering Lists}
\section{4.6. The Peculiar Nature of and and or}
\subsection{4.6.1. Using the and-or Trick}
       
\section{4.7. Using lambda Functions}
\subsection{4.7.1. Real-World lambda Functions}
       
\section{4.8. Putting It All Together}
\section{4.9. Summary}
   
\chapter{5. Objects and Object-Orientation}
\section{5.1. Diving In}
\section{5.2. Importing Modules Using from module import}
\section{5.3. Defining Classes}
\subsection{5.3.1. Initializing and Coding Classes}
\subsection{5.3.2. Knowing When to Use self and \_\_init\_\_}
       
\section{5.4. Instantiating Classes}
\subsection{5.4.1. Garbage Collection}
       
\section{5.5. Exploring UserDict: A Wrapper Class}
\section{5.6. Special Class Methods}
\subsection{5.6.1. Getting and Setting Items}
       
\section{5.7. Advanced Special Class Methods}
\section{5.8. Introducing Class Attributes}
\section{5.9. Private Functions}
\section{5.10. Summary}
   
\chapter{6. Exceptions and File Handling}
\section{6.1. Handling Exceptions}
\subsection{6.1.1. Using Exceptions For Other Purposes}
       
\section{6.2. Working with File Objects}
\subsection{6.2.1. Reading Files}
\subsection{6.2.2. Closing Files}
\subsection{6.2.3. Handling I/O Errors}
\subsection{6.2.4. Writing to Files}
       
\section{6.3. Iterating with for Loops}
\section{6.4. Using sys.modules}
\section{6.5. Working with Directories}
\section{6.6. Putting It All Together}
\section{6.7. Summary}
   
\chapter{7. Regular Expressions}
\section{7.1. Diving In}
\section{7.2. Case Study: Street Addresses}
\section{7.3. Case Study: Roman Numerals}
\subsection{7.3.1. Checking for Thousands}
\subsection{7.3.2. Checking for Hundreds}
       
\section{7.4. Using the {n,m} Syntax}
\subsection{7.4.1. Checking for Tens and Ones}
       
\section{7.5. Verbose Regular Expressions}
\section{7.6. Case study: Parsing Phone Numbers}
\section{7.7. Summary}
   
\chapter{8. HTML Processing}
\section{8.1. Diving in}
\section{8.2. Introducing sgmllib.py}
\section{8.3. Extracting data from HTML documents}
\section{8.4. Introducing BaseHTMLProcessor.py}
\section{8.5. locals and globals}
\section{8.6. Dictionary-based string formatting}
\section{8.7. Quoting attribute values}
\section{8.8. Introducing dialect.py}
\section{8.9. Putting it all together}
\section{8.10. Summary}
   
\chapter{9. XML Processing}
\section{9.1. Diving in}
\section{9.2. Packages}
\section{9.3. Parsing XML}
\section{9.4. Unicode}
\section{9.5. Searching for elements}
\section{9.6. Accessing element attributes}
\section{9.7. Segue}
   
\chapter{10. Scripts and Streams}
\section{10.1. Abstracting input sources}
\section{10.2. Standard input, output, and error}
\section{10.3. Caching node lookups}
\section{10.4. Finding direct children of a node}
\section{10.5. Creating separate handlers by node type}
\section{10.6. Handling command-line arguments}
\section{10.7. Putting it all together}
\section{10.8. Summary}
   
\chapter{11. HTTP Web Services}
\section{11.1. Diving in}
\section{11.2. How not to fetch data over HTTP}
\section{11.3. Features of HTTP}
\subsection{11.3.1. User-Agent}
\subsection{11.3.2. Redirects}
\subsection{11.3.3. Last-Modified/If-Modified-Since}
\subsection{11.3.4. ETag/If-None-Match}
\subsection{11.3.5. Compression}
       
\section{11.4. Debugging HTTP web services}
\section{11.5. Setting the User-Agent}
\section{11.6. Handling Last-Modified and ETag}
\section{11.7. Handling redirects}
\section{11.8. Handling compressed data}
\section{11.9. Putting it all together}
\section{11.10. Summary}
   
\chapter{12. SOAP Web Services}
\section{12.1. Diving In}
\section{12.2. Installing the SOAP Libraries}
\subsection{12.2.1. Installing PyXML}
\subsection{12.2.2. Installing fpconst}
\subsection{12.2.3. Installing SOAPpy}
       
\section{12.3. First Steps with SOAP}
\section{12.4. Debugging SOAP Web Services}
\section{12.5. Introducing WSDL}
\section{12.6. Introspecting SOAP Web Services with WSDL}
\section{12.7. Searching Google}
\section{12.8. Troubleshooting SOAP Web Services}
\section{12.9. Summary}
   
\chapter{13. Unit Testing}
\section{13.1. Introduction to Roman numerals}
\section{13.2. Diving in}
\section{13.3. Introducing romantest.py}
\section{13.4. Testing for success}
\section{13.5. Testing for failure}
\section{13.6. Testing for sanity}
   
\chapter{14. Test-First Programming}
\section{14.1. roman.py, stage 1}
\section{14.2. roman.py, stage 2}
\section{14.3. roman.py, stage 3}
\section{14.4. roman.py, stage 4}
\section{14.5. roman.py, stage 5}
   
\chapter{15. Refactoring}
\section{15.1. Handling bugs}
\section{15.2. Handling changing requirements}
\section{15.3. Refactoring}
\section{15.4. Postscript}
\section{15.5. Summary}
   
\chapter{16. Functional Programming}
\section{16.1. Diving in}
\section{16.2. Finding the path}
\section{16.3. Filtering lists revisited}
\section{16.4. Mapping lists revisited}
\section{16.5. Data-centric programming}
\section{16.6. Dynamically importing modules}
\section{16.7. Putting it all together}
\section{16.8. Summary}
   
\chapter{17. Dynamic functions}
\section{17.1. Diving in}
\section{17.2. plural.py, stage 1}
\section{17.3. plural.py, stage 2}
\section{17.4. plural.py, stage 3}
\section{17.5. plural.py, stage 4}
\section{17.6. plural.py, stage 5}
\section{17.7. plural.py, stage 6}
\section{17.8. Summary}
   
\chapter{18. Performance Tuning}
\section{18.1. Diving in}
\section{18.2. Using the timeit Module}
\section{18.3. Optimizing Regular Expressions}
\section{18.4. Optimizing Dictionary Lookups}
\section{18.5. Optimizing List Operations}
\section{18.6. Optimizing String Manipulation}
\section{18.7. Summary}
      
\appendix
   
\chapter{A. Further reading}
\chapter{B. A 5-minute review}
\chapter{C. Tips and tricks}
\chapter{D. List of examples}
\chapter{E. Revision history}
\chapter{F. About the book}
\chapter{G. GNU Free Documentation License}
\setcounter{section}{-1}
\section{G.0. Preamble}
\section{G.1. Applicability and definitions}
\section{G.2. Verbatim copying}
\section{G.3. Copying in quantity}
\section{G.4. Modifications}
\section{G.5. Combining documents}
\section{G.6. Collections of documents}
\section{G.7. Aggregation with independent works}
\section{G.8. Translation}
\section{G.9. Termination}
\section{G.10. Future revisions of this license}
\section{G.11. How to use this License for your documents}
   
\chapter{H. Python license}

\section{H.A. History of the software}
\section{H.B. Terms and conditions for accessing or otherwise using Python}
\subsection{H.B.1. PSF license agreement}
\subsection{H.B.2. BeOpen Python open source license agreement version 1}
\subsection{H.B.3. CNRI open source GPL-compatible license agreement}
\subsection{H.B.4. CWI permissions statement and disclaimer}

\end{document}
